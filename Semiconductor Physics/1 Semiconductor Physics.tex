\documentclass{article}

\usepackage{amsmath,amssymb}
\usepackage[margin=0.7in]{geometry}
\usepackage{float}
\usepackage{graphicx}

\title{Semiconductor Physics}
\author{Emmanuel Estallo}
\date{\today}

\begin{document}
\boldmath
\maketitle 
\noindent 
\section{Resistivity and Conductivity}
For any material, if we connect two terminals across it, we can obtain its resistance.
The resistance of any device is defined as $$R = \rho \frac{L}{A}$$ where $L$ is the
length of the device, $\rho$ is the resistivity, and $A$ is the cross-sectional area. 
\vspace{8pt}
\\ Similarly, conductivity is obtained by $$G = \frac{1}{R}$$

\section{Quantum Mechanics}
\noindent 
The \textbf{Wave Particle Duality} states that a particle can behave as a wave and at 
the same time, a wave can behave as a particle. This is captured by the 
\textbf{De Broglie Wavelength} which states that for any particle with momentum $p$,
we can associate a wavelength $\lambda$ such that $$\lambda = \frac{h}{p}$$ where $h$
is the Planck's constant.  
\vspace{8pt}
\\ Now, consider the hydrogen atom. It has a single proton, and a single electron on
its orbital. The electron is moving with a momentum $p$ and charge $-e$. If the electron
is treated as a wave, the question is that "\textbf{What are the allowed trajectories?}" 

\vspace{8pt}
\noindent 
From those, we can calculate the allowed energies, and some other parameters. 

\vspace{8pt}
\noindent 
Since the $\psi$ function represents probability of presence, if I want to have a 
significant presence along the orbital of the atom, then we want the circumference 
to be an integer multiple of the wavelength. This means that the integer 
multiples of the wavelength are the possible trajectories. 
$$2\pi r = n\lambda = n \; \frac{h}{p}$$
Is the condition for having a sustainable trajectory. 
\vspace{8pt}
\\ We also know that the electron does not move way from the proton because it is 
bound by a force. That force is $$F = \frac{1}{4\pi\epsilon_{0}} \frac{e^2}{r^2}
= \frac{mv^2}{r} = \frac{p^2}{mr}$$
\newpage 
\noindent 
Simplifying, we get $$2\pi r = n \; \frac{h}{p}$$
$$\frac{1}{4\pi \epsilon_0} \frac{e^2}{r} = \frac{p^2}{m}$$
Obtaining p, $$r = \frac{nh}{2\pi p}$$
$$\frac{1}{4\pi\epsilon_0} \; \frac{e^2 (2\pi p)}{nh} = \frac{p^2}{m} $$
$$\frac{1}{2\epsilon_0} \; \frac{e^2}{nh} = \frac{p}{m}$$
Finally, 
$$\boxed{p = \frac{me^2}{2\epsilon_0 h} \; \frac{1}{n}}$$
Plugging it back to the r equation gives 
$$r = \frac{nh}{2\pi} \; \frac{2\epsilon_0}{me^2} \; nh$$
$$r=  \frac{nh}{\pi} \frac{\epsilon_0}{me^2} \; nh$$
Finally, 
$$\boxed{r = \frac{\epsilon_0}{\pi me^2} \; n^2 h^2}$$
where $n \in \mathbb{N}$ and $m$ is the mass of the electron. 
\vspace{8pt}
\\ This means that both radius and momentum are quantized. They can only take certain 
values given that they have a bound state. It requires that the sum of the kinetic and 
potential energies of the particle be negative. 
$$E_{total} = E_{kinetic} + E_{potential}$$
$$E_{total} = \frac{p^2}{2m} - \frac{1}{4\pi\epsilon_0} \; \frac{e^2}{r} $$ 
$$E_{total} = \frac{p^2}{2m} - \frac{p^2}{m} = -\frac{p^2}{2m} \rightarrow E_{total} < 0$$
Substituting $p$ to the total energy equation gives 
$$E_{total} = -\frac{m e^4}{8\epsilon_0^2 h^2} \cdot \frac{1}{n^2}$$ 
$$\boxed{E_{total} = -\frac{13.6\; eV}{n^2}}$$ 
If we have another atom of the same kind that is close to it, the energy levels of either
of those atoms would change because of the \textbf{Pauli Exclusion Principle.} The 
principle states that you cannot have the same quantum numbers when two particles come
together. 
\vspace{8pt}
\\ Since the two atoms occupy the same spatial area, their energy levels cannot be the 
same. Each energy level will split (each energy level can take two electrons : up-spin
or down-spin). This is what we call degeneracy.
\newpage 
\noindent 
The next question is that: \textbf{When does the Pauli Exclusion Principle become 
important?} Apparently, the principle is always there. No two atoms in the universe have 
the same quantum numbers. It just becomes less noticeable when they are far apart and
more evident when they get close. 
\vspace{8pt}
\\ When we create a crystal lattice, many electrons come together and we see more 
degeneracy. These degenerate energy levels are what we call energy bands. 
\textbf{Hybridization} happens within some of these bands specifically those of 
tetrahedral structure with a $109.5^{o}$ angle. At $0\;K$, the bands are filled from  
the lowest to the highest energy band. Thus for a carbon atom which has 6 electrons,
the $1s^2$ band is filled as well as the $2s^{1}\; 2p^{3}$ hybrid band. 
\vspace{8pt}
\\ Consider two bands, a valence band and a conduction band. If the valence band is full,
then the material is a bad conductor since it allows no electron movement. A good conductor 
is one that allows full electron movement from the valence band to the conduction band.
A \textbf{semiconductor} is somewhat in the middle. The bands where there are no 
electrons can be called holes and carry a positive charge. Electrons move faster than
holes. \textbf{Recombination} is the process of combining holes and electrons within
a band. At the steady-state, the rate of recombination and hole generation are equal.

\section{Solid-State Physics}
\noindent 
The concept of \textbf{Doping} allows us to introduce a different element to control 
the ratio of electrons and holes. For example, if we use an element from the column
$IV$ of the periodic table, then we can choose elements from columns $III$ and $V$ as
\textbf{Dopants.} As we increase the temperature, the number of electrons reach the 
number of dopants since they all become ionized. We call this number $N_D$. Similarly,
we use $N_A$ for the holes. 
\vspace{8pt}
\\ Electrons are fermions. They follow the \textbf{Fermi-Dirac Distribution} which 
tells us that we can obtain the probability of finding an electron with energy $E$ by
$$f(E) = \frac{1}{1+e^{\frac{E-E_{f}}{kT}}}$$ where $k$ is the Boltzmann's constant
and $T$ is the absolute temperature. $E_{f}$ defined as the energy where the probability
of finding an electron is $\frac{1}{2}$. As $E-E_{f}$ gets $>3kT$, the more it 
resembles the \textbf{Boltzmann Distribution.} $$f(E) \propto{e^\frac{-E}{kT}}$$
We can then obtain the relation $$n = N_{c} e^\frac{E_f - E_c}{kT}$$ and 
$$p = N_{v} e^\frac{E_v - E_f}{kT}$$ 
For an intrinsic semiconductor, $$\boxed{n_{i}^2 = np}$$ where $n_i$ is the intrinsic
carrier concentration. 
\vspace{8pt}
\\
Similarly, the density of carriers n and p are given by 
$$\boxed{N_D = n = n_{i}e^{\frac{E_f -E_i}{kT}}}$$
and 
$$\boxed{N_A = p = n_{i}e^{\frac{E_i -E_f}{kT}}}$$

\newpage 
\subsection{Drift and Diffusion}
\noindent 
The \textbf{Equipartition Theorem} states that in thermal equilibrium, each
degree of freedom takes an energy $$E = \frac{1}{2}kT$$ on average. For an electron, 
the degree of freedom is equal to 3. That translates to $$E_{total} = \frac{3}{2}kT$$
From classical mechanics, $$E_{kinetic}=\frac{1}{2}mv^2$$ 
We get, $$\frac{1}{2}mv^2 = \frac{3}{2}kT$$ $$v^2 = \frac{3kT}{m}$$ Since the mass 
is very small for an electron, the velocity is very high. 
\vspace{8pt}
\\ Consider an electric field on which the electron is placed. Assume that the field 
has a magnitude of $E$. 
\vspace{8pt}
\\We know that, $$F = \frac{dP}{dt}$$ and $$-qE = \frac{\Delta P}{\Delta T} 
= \frac{m\Delta v}{\tau_{c}}$$ $$\boxed{\Delta v = \frac{qE \tau_c}{m}}$$ We can assume 
that the velocity, on average, is equal to $$\boxed{v_{ave} = \frac{1}{2} \Delta v = 
\frac{1}{2} \frac{qE \tau_c}{m} = v_{drift}}$$
Which is also equal to $$\boxed{v_{drift} = -\mu_n E}$$ where $\mu_n$ is the mobility.
\vspace{8pt}
\\ The incremental charge that passes through a slice of a slab is obtained by 
$$dq = -qnAv_d \cdot dt$$
where $-q$ is the charge of an electron, $n$ is the electron density, $A$ is the 
cross-sectional area of the slab, and $v_d$ is the drift velocity. We know that 
$$I_n = \frac{dq}{dt} = -qnAv_{d}$$ and $$J_n = \frac{I}{A} =-nqv_d = qn\mu_n E$$
Similarly for holes, $$J_p = qp\mu_p E$$
Finally, $$\boxed{J_{total} = q\cdot(n\mu_n + p\mu_p)E = \sigma E}$$
where $\sigma$ is the conductivity. The resistivity, $\rho$, is the reciprocal of $\sigma$.

\newpage 
\subsection{PN Junction}
\subsubsection{Depletion Region}
\noindent 
Consider two pieces of semiconductor. A P-type semiconductor has acceptor atoms; 
an N-type has donor atoms. In a simple sense, the P-type semiconductor has more holes,
and the N-type semiconductor has more electrons. 
\vspace{8pt}
\\
As the temperature increases, some atoms get ionized. Some holes from the p-type 
material goes to the n-type material. Similarly, some electrons from the n-type 
material goes to the p-type material. The border of these regions get depleted. 
This is what we call the \textbf{Depletion Region.} Recombination happens to some 
electrons and holes. This process happens randomly. 
\vspace{8pt}
\\ This depletion region generates an electric field. As the temeperature increases, 
the depletion region gets larger. As the depletion region gets larger, the generated
electric field becomes stronger. This electric field impedes the movement of a charge
from a region to the other. If an electron has enough energy, then that electron can 
jump from the n region to the p region. Those electron-hole pairs that recombined at 
the depletion region gets affected by the electric field. This opposes the creation of
a depletion region. 
\vspace{8pt}
\\ We know that energy is the product of charge and voltage. Thus,
$$q\psi_p = E_{i,p} - E_{f} = kT \cdot ln\left(\frac{N_A}{n_i}\right)$$
and $$\psi_p = E_{i,p} - E_{f} = \frac{kT}{q} ln\left(\frac{N_A}{n_i}\right)$$
similarly for the electrons, 
$$\psi_n = E_{f} - E_{i,n} = \frac{kT}{q} ln\left(\frac{N_D}{n_i}\right)$$
Finally, 
$$\boxed{\psi_o = \psi_n + \psi_p = \frac{kT}{q}ln\left(\frac{N_A N_D}{n_i^2} \right)}$$ 
where $\psi_o$ is the built in potential. 

\vspace{8pt}
\noindent  
From the Boltzmann distribution, 
$$n \propto \exp\left(-\frac{q\psi_o}{kT}\right)$$
where n is the number of electrons that has enough energy to pass the potential barrier.
This is also true for the holes. 

\vspace{8pt}
\noindent  
If we apply a forward-bias $V_D$ across the ends of the pn junction, then the depletion 
region will be smaller. The electric field will go down and the potential barrier will 
be lowered by $qV_D$. This means that more electrons can pass through. 
$$n \propto \exp\left(-\frac{q(\psi_o -V_D)}{kT}\right)$$ The net flow of electrons is 
obtained by $$\boxed{J_n = \exp\left(-\frac{q(\psi_o -V_D)}{kT}\right)-
\exp\left(-\frac{q\psi_o}{kT}\right) = \exp\left(-\frac{q\psi_o}{kT}\right)\cdot 
\left[\exp\left(\frac{qV_D}{kT}\right)-1\right]}$$
This means that the net current is proportional to
$$\exp\left(\frac{qV_D}{kT}\right)-1$$
The diode current is then obtained by 
$$\boxed{I = I_s \left[\exp\left(\frac{qV_D}{kT}\right)-1\right]}$$ 
where $I_s$ is the proportionality constant. 

\newpage
\subsubsection{Junction Capacitance}
\noindent 
If a reverse bias is applied at the pn junction, the depletion region becomes larger. 
The depletion region behaves like a parallel-plate capacitor. We know that 
$$\frac{C}{A} = \frac{\epsilon}{d}$$
and following the order of integration, $\rho \rightarrow E \rightarrow \psi.$ 
From the volume charge density we can obtain the electric field. From the electric field 
we can obtain the electric potential. Note that the volume charge density $\rho$ is 
equal to the surface charge density $\sigma$ divided by the length. 
$$\rho = \frac{\sigma}{x}$$
\vspace{8pt}
\textbf{Gauss' Law} tells us that the electric field across a gaussian surface is given 
by $$E=\frac{\sigma}{\epsilon}$$
The magnitude of the electric field at the depletion region is given by
$$\boxed{E_{max} = \frac{qx_nN_D}{\epsilon} = \frac{qx_pN_A}{\epsilon}}$$ where $x_n$ and $x_p$
are the lengths of the depletion region on the n and p regions respectively. 
\vspace{8pt}
\\ Since we started with total charge neutrality, the net charges should be equal. 
$$x_nN_D = x_pN_A$$
$$x_n + x_p = x_n + x_n \frac{N_D}{N_A} = x_n \left(\frac{N_A + N_D}{N_A}\right)$$
$$\boxed{x_n N_D = \frac{N_A N_D}{N_A + N_D}(x_n + x_p)}$$
Thus, $$E_{max}= \frac{q}{\epsilon} \frac{N_A N_D}{N_A + N_D}(x_n + x_p)$$
The potential on the p and n sides of the depletion region are given by 
$$\psi_p = \frac{1}{2}x_p E_{max}$$ $$\psi_n = \frac{1}{2}x_n E_{max}$$
We know that applying an external reverse bias $-V_D$ gives 
$$\psi_o - V_D = \psi_p + \psi_n = \frac{E_{max}}{2}(\psi_p + \psi_n)
= \frac{q}{2\epsilon}\left(\frac{N_A N_D}{N_A + N_D}\right)(x_n + x_p)^2$$
$$x_n + x_p = \sqrt{\frac{2\epsilon}{q}\left(\frac{1}{N_A}+\frac{1}{N_D}\right) 
(\psi_o - V_D)}$$
Finally, 
$$\boxed{C_j = \sqrt{\frac{q\epsilon}{2\left(\frac{1}{N_A}+\frac{1}{N_D}\right)
(\psi_o -V_D)}} = \frac{C_{j0}}{\left(1-\frac{V_D}{\psi_o}\right)^\frac{1}{2}}}$$
Where $C_{j0}$ is the value of $C_j$ at $V_D = 0$.
\end{document}