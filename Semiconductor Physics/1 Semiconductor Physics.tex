\documentclass{article}

\usepackage{amsmath,amssymb}
\usepackage[margin=0.7in]{geometry}

\title{Semiconductor Physics}
\author{Emmanuel Estallo}
\date{\today}

\begin{document}
\boldmath
\maketitle 
\noindent 
\section{Resistivity and Conductivity}
For any material, if we connect two terminals across it, we can obtain its resistance.
The resistance of any device is defined as $$R = \rho \frac{L}{A}$$ where $L$ is the
length of the device, $\rho$ is the resistivity, and $A$ is the cross-sectional area. 
\vspace{8pt}
\\ Similarly, conductivity is obtained by $$G = \frac{1}{R}$$

\section{Quantum Mechanics}
\noindent 
The \textbf{Wave Particle Duality} states that a particle can behave as a wave and at 
the same time, a wave can behave as a particle. This is captured by the 
\textbf{De Broglie Wavelength} which states that for any particle with momentum $p$,
we can associate a wavelength $\lambda$ such that $$\lambda = \frac{h}{p}$$ where $h$
is the Planck's constant.  
\vspace{8pt}
\\ Now, consider the hydrogen atom. It has a single proton, and a single electron on
its orbital. The electron is moving with a momentum $p$ and charge $-e$. If the electron
is treated as a wave, the question is that "\textbf{What are the allowed trajectories?}" 

\vspace{8pt}
\noindent 
From those, we can calculate the allowed energies, and some other parameters. 

\vspace{8pt}
\noindent 
Since the $\psi$ function represents probability of presence, if I want to have a 
significant presence along the orbital of the atom, then we want the circumference 
to be an integer multiple of the wavelength. This means that the integer 
multiples of the wavelength are the possible trajectories. 
$$2\pi r = n\lambda = n \; \frac{h}{p}$$
Is the condition for having a sustainable trajectory. 
\vspace{8pt}
\\ We also know that the electron does not move way from the proton because it is 
bound by a force. That force is $$F = \frac{1}{4\pi\epsilon_{0}} \frac{e^2}{r^2}
= \frac{mv^2}{r} = \frac{p^2}{mr}$$
\newpage 
\noindent 
Simplifying, we get $$2\pi r = n \; \frac{h}{p}$$
$$\frac{1}{4\pi \epsilon_0} \frac{e^2}{r} = \frac{p^2}{m}$$
Obtaining p, $$r = \frac{nh}{2\pi p}$$
$$\frac{1}{4\pi\epsilon_0} \; \frac{e^2 (2\pi p)}{nh} = \frac{p^2}{m} $$
$$\frac{1}{2\epsilon_0} \; \frac{e^2}{nh} = \frac{p}{m}$$
Finally, 
$$\boxed{p = \frac{me^2}{2\epsilon_0 h} \; \frac{1}{n}}$$
Plugging it back to the r equation gives 
$$r = \frac{nh}{2\pi} \; \frac{2\epsilon_0}{me^2} \; nh$$
$$r=  \frac{nh}{\pi} \frac{\epsilon_0}{me^2} \; nh$$
Finally, 
$$\boxed{r = \frac{\epsilon_0}{\pi me^2} \; n^2 h^2}$$
where $n \in \mathbb{N}$ and $m$ is the mass of the electron. 
\vspace{8pt}
\\ This means that both radius and momentum are quantized. They can only take certain 
values given that they have a bound state. It requires that the sum of the kinetic and 
potential energies of the particle be negative. 
$$E_{total} = E_{kinetic} + E_{potential}$$
$$E_{total} = \frac{p^2}{2m} - \frac{1}{4\pi\epsilon_0} \; \frac{e^2}{r} $$ 
$$E_{total} = \frac{p^2}{2m} - \frac{p^2}{m} = -\frac{p^2}{2m} \rightarrow E_{total} < 0$$
Substituting $p$ to the total energy equation gives 
$$E_{total} = -\frac{m e^4}{8\epsilon_0^2 h^2} \cdot \frac{1}{n^2}$$ 
$$\boxed{E_{total} = -\frac{13.6\; eV}{n^2}}$$
\newpage 
\noindent 
If we have another atom of the same kind that is close to it, the energy levels of either
of those atoms would change because of the \textbf{Pauli Exclusion Principle.} The 
principle states that you cannot have the same quantum numbers when two particles come
together. 
\vspace{8pt}
\\ Since the two atoms occupy the same spatial area, their energy levels cannot be the 
same. Each energy level will split (each energy level can take two electrons : up-spin
or down-spin). This is what we call degeneracy.
\vspace{8pt}
\\The next question is that: \textbf{When does the Pauli Exclusion Principle become 
important?} Apparently, the principle is always there. No two atoms in the universe have 
the same quantum numbers. It just becomes less noticeable when they are far apart and
more evident when they get close. 
\vspace{8pt}
\\ When we create a crystal lattice, many electrons come together and we see more 
degeneracy. These degenerate energy levels are what we call energy bands. 
\textbf{Hybridization} happens within some of these bands specifically those of 
tetrahedral structure with a $109.5^{o}$ angle. At $0\;K$, the bands are filled from  
the lowest to the highest energy band. Thus for a carbon atom which has 6 electrons,
the $1s^2$ band is filled as well as the $2s^{1}\; 2p^{3}$ hybrid band. 
\vspace{8pt}
\\ Consider two bands, an valence band and a conduction band. If the valence band is full,
then the material is a bad conductor since it allows no electron movement. A good conductor 
is one that allows full electron movement from the valence band to the conduction band.
A \textbf{semiconductor} is somewhat in the middle. The bands where there are no 
electrons can be called holes and carry a positive charge. Electrons move faster than
holes. \textbf{Recombination} is the process of combining holes and electrons within
a band. At the steady-state, the rate of recombination and hole generation are equal.

\section{Solid-State Physics}
\noindent 
The concept of \textbf{Doping} allows us to introduce a different element to control 
the ratio of electrons and holes. For example, if we use an element from the column
$IV$ of the periodic table, then we can choose elements from columns $III$ and $V$ as
\textbf{Dopants.} As we increase the temperature, the number of electrons reach the 
number of dopants since they all become ionized. We call this number $N_D$. Similarly,
we use $N_A$ for the holes. 
\vspace{8pt}
\\ Electrons are fermions. They follow the \textbf{Fermi-Dirac Distribution} which 
tells us that we can obtain the probability of finding an electron with energy $E$ by
$$f(E) = \frac{1}{1+e^{\frac{E-E_{f}}{kT}}}$$ where $k$ is the Boltzmann's constant
and $T$ is the absolute temperature. $E_{f}$ defined as the energy where the probability
of finding an electron is $\frac{1}{2}$. As $E-E_{f}$ gets $>3kT$, the more it 
resembles the Boltzmann distribution. $$f(E) \propto{e^\frac{-E}{kT}}$$
We can then obtain the relation $$n = N_{c} e^\frac{E_f - E_c}{kT}$$ and 
$$p = N_{v} e^\frac{E_v - E_f}{kT}$$ 
For an intrinsic semiconductor, $$\boxed{n_{i}^2 = np}$$

\newpage 
\section{Drift and Diffusion}
\noindent 





















\end{document}