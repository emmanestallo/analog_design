\documentclass{article}

\usepackage{amsmath,amssymb}
\usepackage[margin=0.7in]{geometry}
\usepackage{float}
\usepackage{graphicx}

\title{Transistors}
\author{Emmanuel Estallo}

\begin{document}
\boldmath 
\maketitle 

\section{Bipolar Junction Transistor}
\noindent 
The behavior of the majority carriers are dictated by the behavior of the minority.
If we let the minority carriers propagate over time, eventually all of them will 
recombine. The goal is to manipulate these minority carriers in a region where there 
is no significant amount of recombination. Basically, we want to get these minority 
carriers to a place where they are the majority carriers again. 

\subsection{Basic Operation}
\noindent 
Adding an n region beside the p region of a np junction gives us an npn transistor.
An external voltage source connected to the pn regions lowers the potential barrier.
In a sense, the added n region is used to collect the emitted electrons from the first 
n region. 
\vspace{8pt}
\\ Since there are two pn junctions, there will be two depletion regions with electric
fields both pointing to the p region. For the \textbf{Base-Emitter} junction, we should 
apply a forward voltage bias $V_{BE}$ to lower the potential barrier and decrease the 
strength of electric field that is generated by the depletion region. For the 
\textbf{Base-Collector} junction, we should apply a reverse voltage bias $V_{BC}$ 
to increase the electric field strength. This allows electrons to pass through easier. 
\vspace{8pt}
\\ We want to get the electrons from the emitter region to the collector. Thus, it will
be better to make the base region as thin as possible. Because of this, the 
concentration of the minority carriers decrease immediately. This resembles the diode.
The collector current is obtained by 
$$I_{C}=I_{S}\exp \left(\frac{V_{BE}}{V_{T}}\right)$$
\noindent 
We want to make the base very thin so that more electrons can pass through easier. If
we do this, the depletion regions might overlap. As a general rule, to scale transistors
smaller, we should adjust the doping concentrations on all sides. This allows us to 
scale the depletion regions at a desired ratio. This also decreases the resistivity at
the base. The bigger problem is that more holes will get to the emitter. 
\vspace{8pt}
\\ We define the emitter injection efficiency as 
$$\gamma = \frac{I_{nE}}{I_{nE}+I_{pE}}$$
Where $I_{nE}$ is the emitter current carried by electrons and $I_{pE}$ is the emitter
current carried by the holes. We want this efficiency to be close to $100\%$. A way to 
do this is by heavily doping the emitter and lightly doping the base. Another is that 
we can use a different semiconductor material. This semiconductor material will have 
a different bandgap. This is the foundation of \textbf{Heterojunction Bipolar Transistors.}
\vspace{8pt}
\\
Define the base transfer factor as $$\alpha_{T} = \frac{I_{nC}}{I_{nE}}$$
$$\alpha_0 = \frac{I_C}{I_E}$$

\noindent 
This is the current that goes to the collector from the emitter. 

\newpage
\subsection{Early Effect and Dynamics}




\end{document}