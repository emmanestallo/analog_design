\documentclass{article}

\usepackage{amsmath,amssymb}
\usepackage[margin=0.7in]{geometry}
\usepackage{float}
\usepackage{graphicx}

\title{Transistors}
\author{Emmanuel Estallo}
\date{\today}

\begin{document}
\boldmath 
\maketitle 

\section{Bipolar Junction Transistor}
\noindent 
The behavior of the majority carriers are dictated by the behavior of the minority.
If we let the minority carriers propagate over time, eventually all of them will 
recombine. The goal is to manipulate these minority carriers in a region where there 
is no significant amount of recombination. Basically, we want to get these minority 
carriers to a place where they are the majority carriers again. 

\subsection{Basic Operation}
\noindent 
Adding an n region beside the p region of a np junction gives us an npn transistor.
An external voltage source connected to the pn regions lowers the potential barrier.
In a sense, the added n region is used to collect the emitted electrons from the first 
n region. 
\vspace{8pt}
\\ Since there are two pn junctions, there will be two depletion regions with electric
fields both pointing to the p region. For the \textbf{Base-Emitter} junction, we should 
apply a forward voltage bias $V_{BE}$ to lower the potential barrier and decrease the 
strength of electric field that is generated by the depletion region. For the 
\textbf{Base-Collector} junction, we should apply a reverse voltage bias $V_{BC}$ 
to increase the electric field strength. This allows electrons to pass through easier. 
\vspace{8pt}
\\ We want to get the electrons from the emitter region to the collector. Thus, it will
be better to make the base region as thin as possible. Because of this, the 
concentration of the minority carriers decrease immediately. This resembles the diode.
The collector current is obtained by 
$$I_{C}=I_{s}\exp{\frac{V_{BE}}{V_{T}}}$$
essentially, it is a voltage controlled current source. 
 


\end{document}